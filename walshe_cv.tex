\documentclass[11pt,a4paper]{article}
\usepackage[margin=1in]{geometry}
\usepackage{enumitem}
\usepackage{titlesec}
\usepackage{xcolor}
\usepackage{fontspec}
\usepackage{hyperref}
\usepackage{fancyhdr}
\usepackage{multicol}
\usepackage{graphicx}
\usepackage{amsmath}
\usepackage{amsfonts}

% Color definitions
\definecolor{darkblue}{RGB}{25,25,112}
\definecolor{mediumgray}{RGB}{128,128,128}

% Hyperlink setup
\hypersetup{
    colorlinks=true,
    linkcolor=darkblue,
    urlcolor=darkblue,
    citecolor=darkblue
}

% Custom section formatting
\titleformat{\section}
{\Large\bfseries\color{darkblue}}
{}{0em}
{}[\titlerule]

\titleformat{\subsection}
{\large\bfseries}
{}{0em}
{}

% Custom list formatting
\setlist[itemize]{leftmargin=*,topsep=2pt,parsep=0pt,itemsep=2pt}
\setlist[enumerate]{leftmargin=*,topsep=2pt,parsep=0pt,itemsep=2pt}

% Remove page numbering
\pagestyle{empty}

% Custom commands
\newcommand{\sepline}{\noindent\rule{\textwidth}{0.5pt}\vspace{0.5em}}

\begin{document}

% Header
\begin{center}
    {\Huge\bfseries\color{darkblue} CALEN WALSHE}\\[0.3em]
    {\large Staff Quantitative Researcher}\\[0.5em]

    \href{mailto:calen.walshe@gmail.com}{\texttt{calen.walshe@gmail.com}} $\mid$
    \href{https://www.calenwalshe.com}{\texttt{www.calenwalshe.com}} $\mid$
    \href{https://github.com/calenwalshe}{\texttt{github.com/calenwalshe}} $\mid$
    \href{https://www.linkedin.com/in/calen-walshe}{\texttt{linkedin.com/in/calen-walshe}}\\[0.2em]

    {\color{mediumgray} Seattle / Redmond, WA}
\end{center}

\vspace{0.5em}
\sepline

% Summary Section
\section*{SUMMARY}
Calen Walshe is a survey scientist and applied ML researcher who builds end-to-end systems for turning human feedback into production signals. He combines classical survey methodology, causal inference, and NLP to derive insights from implicit user behavior. At Meta, he leads the design and deployment of scalable pipelines that feed quality metrics and sentiment signals directly into core ranking and recommendation models. He is deeply attuned to measurement reliability, drift detection, and end-to-end observability in real-time serving environments. A data professional who ships, Calen bridges behavioral science and infrastructure—from conceptual survey design to engineering high-performance inference systems.

\vspace{0.5em}
\sepline

% Core Competencies Section
\section*{CORE COMPETENCIES}

\subsection*{System Design \& Data Architecture}
\begin{itemize}
    \item Design end-to-end pipelines converting human feedback into production-ready ML signals
    \item Architect scalable survey collection systems with real-time quality validation mechanisms
    \item Build data lakes optimized for behavioral signals, text analysis, and inference workflows
    \item Engineer schema standards that bridge research prototypes and production ranking models
\end{itemize}

\subsection*{Causal Inference \& Measurement Science}
\begin{itemize}
    \item Design randomized controlled trials with statistical power for product-scale intervention measurement
    \item Implement quasi-experimental methods for observational inference when randomization is infeasible
    \item Validate measurement instruments through reliability testing and sensitivity analysis frameworks
\end{itemize}

\subsection*{Behavioral Signal Engineering}
\begin{itemize}
    \item Transform open-text survey responses into structured features using NLP and embedding techniques
    \item Extract implicit sentiment signals from behavioral data patterns and engagement metrics
    \item Design composite quality scores that correlate with long-term user satisfaction outcomes
\end{itemize}

\subsection*{ML Deployment \& Observability}
\begin{itemize}
    \item Deploy production inference systems with sub-millisecond latency requirements using distributed computing
    \item Implement comprehensive monitoring for model drift, data quality, and prediction accuracy
    \item Build reproducible training pipelines with version control for features, models, and evaluation metrics
    \item Design A/B testing frameworks for online model evaluation and gradual deployment strategies
\end{itemize}

\subsection*{Strategic Influence \& Knowledge Systems}
\begin{itemize}
    \item Create decision frameworks that translate complex behavioral research into actionable product strategy
    \item Develop internal methodology playbooks for scaling survey science across product verticals
    \item Mentor cross-functional teams on measurement design, statistical interpretation, and causal reasoning
    \item Build knowledge-sharing systems that democratize access to behavioral insights and research methods
\end{itemize}

\vspace{0.5em}
\sepline

% Experience Section
\section*{EXPERIENCE}

\subsection*{Meta --- Staff Quantitative Researcher}
\textit{Nov 2021 -- Present $\mid$ Bellevue, WA}
\begin{itemize}
    \item Led quantitative UX research programs measuring ad surface usability and user sentiment by integrating in-product telemetry with high-volume survey instrumentation to pinpoint friction points.
    \item Ran large-scale UX experiments (randomized and quasi-experimental) to evaluate design and policy changes on satisfaction; results steered multi-billion-dollar roadmap decisions.
    \item Defined user experience north-star metrics that translate qualitative insights into production ranking signals; adopted as standards across product and engineering teams.
    \item Built production SQL and Python pipelines that stitch survey feedback with interaction traces, enabling self-serve UX analytics and longitudinal cohort tracking.
    \item Deployed LLM-based classifiers to code millions of open-ended responses and surface emergent UX issues in real time for ranking and policy partners.
    \item Created dashboards aligning design, PM, and engineering stakeholders on journey health metrics; now core to weekly decision rituals.
    \item Partnered with design, policy, and trust teams to embed UX guardrails in feature launches, balancing experience, safety, and implementation cost.
\end{itemize}

\subsection*{C. Light Technologies --- Data Scientist}
\textit{Mar 2021 -- Nov 2021 $\mid$ Berkeley, CA}
\begin{itemize}
    \item Built software classifiers to automatically detect poor-quality retinal scans, improving diagnostic reliability.
    \item Collaborated with hardware engineers to embed the detection system directly into devices, enabling real-time rejection of invalid scans.
    \item Work directly contributed to patent US20250057414A1 on retinal disease detection methods (\href{https://patents.google.com/patent/US20250057414A1/}{link}).
\end{itemize}

\subsection*{Center for Perceptual Systems --- Research Scientist (Postdoc)}
\textit{Aug 2015 -- Mar 2021 $\mid$ Austin, TX}
\begin{itemize}
    \item Designed and executed computational models of human perception using Bayesian inference, machine learning, and large-scale neuroimaging data.
    \item Built reproducible analysis pipelines and simulation frameworks to evaluate cognitive performance and visual attention under uncertainty.
    \item Delivered first-author research featured in \textit{Current Biology} and AAAI, translating findings into algorithms for perception and decision modeling.
    \item Partnered across neuroscience, psychology, and computer science to prototype ML-driven methods and secure multi-disciplinary research funding.
\end{itemize}

\vspace{0.5em}
\sepline

% Education Section
\section*{EDUCATION}

\subsection*{PhD, Cognitive Neuroscience --- University of Edinburgh}
\textit{2011--2015}\\
Computational Visual Cognition Lab $\cdot$ Dean's Scholarship \& NSERC Fellowship

\subsection*{BSc, Cognitive Science (AI concentration) --- Simon Fraser University}
\textit{2005--2009}

\vspace{0.5em}
\sepline

% Publications Section
\section*{SELECTED PUBLICATIONS}
\begin{itemize}
    \item \textbf{Walshe, R.C.}, Geisler, W.S. (2022). Efficient Allocation of Attentional Sensitivity Gain in Visual Cortex Reduces Foveal Sensitivity in Visual Search. \textit{Current Biology}.
    \item Zhang, R., \textbf{Walshe, R.C.}, Liu, Z., Guan, L., Muller, K.S., Writner, J.A., Zhang, L., Hayhoe, M.M., Ballard, D.H. (2020). Atari-HEAD: Atari Human Eye-Tracking and Demonstration Dataset. \textit{Proceedings of the Thirty-Fourth AAAI Conference on Artificial Intelligence (AAAI-20)}.
    \item \textbf{Walshe, R.C.}, Nuthmann, A. (2021). A computational dual-process model of fixation duration control in natural scene viewing. \textit{Computational Brain \& Behavior}.
\end{itemize}

\end{document}
